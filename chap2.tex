\chapter{Types of Malware\label{chap:malware type}}

\vspace{4mm}
Malware refers to any software that intends to damage or disable computers or computer systems. There are many types of malware including but not limited to Adware, Spyware, Ransomware, Virus, Worm, Trojan, Rootkit, Backdoors, Keyloggers, Rogue security software, Browser hijacker. In this chapter, we will look into different types of malware based on action, concealment strategy and malware detection techniques.
\\
\section{Types of Malware - Based on Action, Concealment Strategy}

\textbf{Adware} displays ads on your computer and generates revenue based on that for the malware writer. This is mostly innocuous but might be irritating in that it might open multiple browser windows to display multiple ads at the same time.

\textbf{Spyware} is a kind of malware that tracks your online presence in order to send targeted ads to your computer. For example, say if you are looking for cars, then a car company might use this data to send you ads about their new cars.

\textbf{Ransomware} is a type of malware that encrypts data in infected system or alters its access privileges or blocks access to the computer until a sum of money is paid. This is currently used against individual computers most of the time.

\textbf{Virus} is a self replicating program that mostly attach themselves to other programs. They try to infect their targets and conceal themselves, they are parasitic by nature and reside in the boot sector, executables or data files. They are usually spread by sharing files and portable memory sticks between computers.

\textbf{Worms} are programs that replicate and multiply and repeat this process over and over again thereby effectively choking the system and its resources. They are usually spread through the network.

\textbf{Trojan} are written with the purpose of discovering your financial information, taking over your computer's system resources, and in larger systems creating a denial-of-service attack. They are usually disguised as some other innocuous file like images or benign executables or videos but if you look close enough you will find that the extension differs.

\textbf{Rootkit} is a type of malware that resides in the boot sector and as a result is amongst the first programs to start in the machine when it gets power. It is amongst the hardest of all malware to detect and therefore to remove. It is designed to permit other information gathering malware to get information from your computer without you realizing anything is going on.

\textbf{Backdoor} works by bypassing authentication and gaining remote access to the infected system. It usually waits for commands from a command and control center. Harebot.M is an example of a backdoor malware.

\textbf{Keyloggers} are designed to conceal themselves and keep running in the background and record, transmit every key stroke being executed on the machine. Nowadays they are advanced enough to capture screens and transmit them securely over to a remote server.

\textbf{Rogue security software} impersonate an anti-virus product. They often turn off the real anti-virus products and trick you into believing that these are legitimate programs.

\textbf{Browser hijacker} is designed to route you to pages that would bring money to the malware writers. They work by redirecting the browser traffic to pages of the malware writer's choice.


\section{Types of Malware - Based on Evasion Strategy}

In order to evade detection, malware writers have come up with various effective techniques, we will be seeing some of those techniques below. 

\textbf{Encrypted Virus} work by encrypting the critical code, so that they do not show up on the signature scan. When the virus executes based on some predefined condition the decryption code kicks in and decrypts the critical part at runtime. However, the problem is that the decryption part would be fairly standard and that could be used for scanning the signatures.

\textbf{Oligomorphic Virus} is an improvement over the encrypted virus. They try to offset the weakness in the encrypted virus by having multiple versions of the decryption code. But this will not be that detrimental to the anti-virus product as it will simply add all versions of the decryption code to their databases.

\textbf{Polymorphic virus} is written by encapsulating the layer of encryption and the decryption code is programmed in such a way that it changes its form and hence evades detection. These can be detected by allowing them to run in an emulator and let the code decrypt itself and after that looking for the signature.

\textbf{Metamorphic virus} are the most advanced of all malware. They work by changing the entire code every time they are run. They usually have a mutation engine which serves up several obfuscation techniques to evade signature detection. Since this is the most advanced of all malware it is also the most difficult of malware's to write. This is a challenging problem for anti-virus companies and researchers alike.
\\
\section{Malware Detection Methods}

There are two traditional ways for malware detection, they are:

1) Static Analysis
2) Dynamic Analysis

\textbf{Static Analysis} of malware refers to the slew of methods that extract features from a malware program without actually executing it. These features include but are not limited to opcodes, binary representation of the program, function names, etc. The antivirus program then tries to match a signature from its known database with these static features of the malware. If a match is found, then it is flagged as a malware. There are efficient string comparison algorithms like Aho-Corasick \cite{Bayer2006}, \cite{4317914} which make this process possible. However, this approach could be defeated by using various code obfuscation techniques. We will discuss some of these ways in the next chapter.

\textbf{Dynamic Analysis} of malware refers to techniques in which the malware is run in a sandboxed environment and various features such as windows API calls, stack contents, actual function parameters, etc are extracted from it which is then used for classifying it as a malware. Dynamic analysis is costlier than static analysis but genereally yields better performance. Though it is harder to defeat it has been shown that it is possible to defeat it, we will discuss some of these techniques in the next chapter.
