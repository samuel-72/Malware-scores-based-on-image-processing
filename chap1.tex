\chapter{Introduction\label{chap:introduction}}

\vspace{4mm}
When computers became ubiquitous so did malware. Malware is software that is intended to damage or disable computers and computer systems. Malware writers started out writing code to show off their skills or for research purposes but that soon took a turn for the worse when computer systems started becoming the cornerstone of all sorts of businesses and organizations. Nowadays, most malware are targeted at profit through gaining access to confidential data or destruction of data or denial of services. Hence, it is important to develop efficient methods that could shield the computer systems from these malware.
\\
\\
Classification of malware is a difficult problem to solve on account of the huge number of variants of malware being generated every day. In 2010, Symantec reported that it has over 286 million variants of malware in its corpus \cite{vb}. Traditionally there are two approaches to malware analysis, static and dynamic analysis. Static analysis depends primarily on signature based detection of binary code \cite{ranveer_hiray_2015} or features like opcodes that could be extracted without executing the malware. In dynamic analysis the malware program is allowed to execute under a controlled environment and detection is based on Application Program Interface (API) calls, pattern of memory access, etc. It has been shown that both these approaches have their drawbacks \cite{moser_kruegel_kirda_2007} \cite{sharif2008impeding} \cite{5633410} \cite{Bayer2006}, some of which we will discuss in the next two sections. There is a need for a new robust mechanism for malware classification it is this gap that this research tries to address.
\\
\\
In this research, we will be using techniques from image similarity detection domain. The principal idea is to convert malware binaries in to images and then use gist descriptors for identifying similar images. This technique would help us to identify variants of the same family as they tend to be visually similar even though there has been significant code obfuscation. The research also looks at various techniques for selecting only the most important features.
\\
\\
This paper is structured as follows. Chapter 2 provides details on types of malware. Chapter 3 provide details on drawbacks of static and dynamic analysis respectively. Chapter 4 discusses about using gist features for image recognition and feature selection techniques for reducing the dimensionality of the data. Chapter 5 concludes the paper after discussing about Future Work.
